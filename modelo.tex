
\documentclass[tg]{mdtufsm}


\usepackage{caption}

\captionsetup{
	justification=raggedright,
	singlelinecheck=false
}

\usepackage{ragged2e}
\usepackage{colortbl}
\usepackage{array,multirow}
\usepackage{float} 
\usepackage[portuguese,titlenumbered,ruled]{algorithm2e} % algoritmos
\usepackage[T1]{fontenc}        % pacote para conj. de caracteres correto
\usepackage{fix-cm} %para funcionar corretamente o tamanho das fontes da capa
\usepackage{times, color, xcolor}       % pacote para usar fonte Adobe Times e cores
\usepackage[utf8]{inputenc}   % pacote para acentuação
\usepackage{graphicx}  % pacote para importar figuras
\usepackage{amsmath,latexsym,amssymb} %Pacotes matemáticos
\usepackage[%hidelinks%, 
            bookmarksopen=true,linktocpage,colorlinks=true,
            linkcolor=black,citecolor=black,filecolor=magenta,urlcolor=blue,
            pdftitle={Título do Trabalho},
            pdfauthor={nome do autor},
            pdfsubject={Trabalho de Conclusão de Curso},
            pdfkeywords={modelo, latex, tcc, graduação}
            ]{hyperref} 

%Margens conforme MDT 2015
\usepackage[inner=30mm,outer=20mm,top=30mm,bottom=20mm]{geometry} 

%==============================================================================
% Se o pacote hyperref foi carregado a linha abaixo corrige um bug na hora
% de montar o sumário da lista de figuras e tabelas
% Se o pacote não foi carregado, comentar a linha %
%==============================================================================
\input{macros/bugcaption}

%==============================================================================
% Identificação do trabalho
%==============================================================================
\title{Título do Trabalho}

\author{Sobre nome}{nome do autor}
%Descomentar se for uma "autora"
%\autoratrue


%folha rosto
\course{Curso Superior de Tecnologia em Redes de Computadores}
\institute{Colégio Técnico Industrial de Santa Maria}
\degree{Tecnólogo em Redes de Computadores}

% Número do TG (verificar na secretaria do curso)
% Para mestrado deixar sem opção dentro do {}
\trabalhoNumero{}

%Orientador
\advisor[Prof.]{Dr. (UFSM)}{Sobre nome}{Nome do orientador}
%Se for uma ``orientadora'' descomentar a linha baixo
%\orientadoratrue

%Co orientador, comentar se não existir
%\coadvisor[Prof.]{Drª.}{Pereira}{Maria Regina}
%\coorientadoratrue %Se for uma ``Co-Orientadora''

%Avaliadores (Banca)
\committee[Me.]{Sobre nome}{Nome menbro banca}{UFSM}
\committee[Tecg.]{Sobre nome}{Nome menbro banca}{UFSM}

% a data deve ser a da defesa; 
% dia, mes e ano correntes
\date{dia}{mês}{ano} 

%Palavras chave
\keyword{modelo}
\keyword{latex} 
\keyword{tcc}
\keyword{graduação}

%keyword
\keywordAbstract{model}
\keywordAbstract{latex}
\keywordAbstract{tcc}
\keywordAbstract{graduation}
%%=============================================================================
%% Início do documento
%%=============================================================================
\begin{document}

%%=============================================================================
%% Capa e folha de rosto
%%=============================================================================
\maketitle
	
%%=============================================================================
%folha de aprovação
%%=============================================================================
\makeapprove

%%=============================================================================
%% Dedicatória (opcional)
%%=============================================================================
\chapter*{Dedicatória}


\vfill

\begin{center}
	
	{\sffamily\itshape Dedico este trabalho a ......}
	
\end{center}

\vfill

\newpage
%%=============================================================================
%% Agradecimentos (opcional)
%%=============================================================================
\chapter*{Agradecimentos}


\justifying{\textit{Agradeço .....}}



%%=============================================================================
%% Epígrafe (opcional)
%%=============================================================================
\clearpage
\begin{flushright}
\mbox{}\vfill

{\sffamily\itshape
	\begin{quote}
		\begin{quote}
			\begin{quotation}
			``O valor de uma coisa depende da maneira como a abordamos mentalmente e não da coisa em si'' \\
			
			\begin{flushright}
			\textsc{(Jigoro Kano)}
		  	\end{flushright}
		  	
		\end{quotation}
		\end{quote}
	\end{quote}
 }




\end{flushright}

%%=============================================================================
%% Resumo
%%=============================================================================
\begin{abstract}
		
Aqui você escreve o resumo. Lembrando no máximo 250 palavras.

\end{abstract}

%%=============================================================================
%% Abstract
%%=============================================================================
% resumo na outra língua
% como parametros devem ser passados o titulo, o nome do curso,
% as palavras-chave na outra língua, separadas por vírgulas, o mês em inglês
%o a sigla do dia em inglês: st, nd, th ...
\begin{englishabstract}
{Abstract title}
%%============ MANTER ESSA LINHA ============================================
\ %%============ MANTER ESSA LINHA ==========================================
%%============ MANTER ESSA LINHA ============================================
\ %%============ MANTER ESSA LINHA ==========================================
%%============ MANTER ESSA LINHA ============================================
\ %%============ MANTER ESSA LINHA e deixar uma em branco ===================

Here you write the summary. Remembering a maximum of 250 words.

\end{englishabstract}

%% Lista de Ilustrações (opc)
%% Lista de Símbolos (opc)
%% Lista de Anexos e Apêndices (opc)

%%=============================================================================
%% Lista de figuras (comentar se não houver)
%%=============================================================================
\listoffigures
%%=============================================================================
%% Lista de tabelas (comentar se não houver)
%%=============================================================================
\listoftables

%%=============================================================================
%% Lista de Apêndices (comentar se não houver)
%%=============================================================================
%\listofappendix

%%=============================================================================
%% Lista de Anexos (comentar se não houver)
%%=============================================================================
%\listofannex

%%=============================================================================
%% Lista de abreviaturas e siglas
%%=============================================================================
 %o parametro deve ser a abreviatura mais longa

\begin{listofabbrv}{UbiComp}
	

	  \item [MDT]       Manual de Dissertação e Tese
	  \item [UFSM]      Universidade Federal de Santa Maria


\end{listofabbrv}

%%=============================================================================
%% Lista de simbolos (opcional)
%%=============================================================================
%Simbolos devem aparecer conforme a ordem em que aparecem no texto
% o parametro deve ser o símbolo mais longo
%\begin{listofsymbols}{teste}
%  \item [$\varnothing$] vazio
 % \item [$\Gamma$]  Gama
%  \item [$\forall$] Para todo
%\end{listofsymbols}

%%=============================================================================
%% Sumário
%%=============================================================================
\tableofcontents


%%=============================================================================
%% Início da dissertação
%%=============================================================================
\setlength{\baselineskip}{1.5\baselineskip}

%Adiciona cada capitulo
\chapter{Introdução}
\label{chap:Introducao}

Está é a introdução do trabalho. Este modelo estar formatado conforme a MDT (Manual de Dissertação e Tese)da UFSM (Universidade Federal de Santa Maria) 20015. 

Um bom livro de linguagens de programação é o \cite{Sebesta:2005}. 
Conforme Sebesta \citeyearpar{Sebesta:2005}, uma boa linguagem de programação é Java \cite{Sun:2010}.

Segundo Lee~\citeyearpar{Lee:2009}, a definição de contexto mais citada na bibliografia é a definição
proposta por Abowd \textit{et al.}:

\begin{quote}
         Contexto é qualquer informação que pode ser utilizada 
         para caracterizar a situação de uma entidade. Uma entidade é uma pessoa, lugar ou objeto que 
         podem ser considerados relevantes para a interação entre um usuário e uma aplicação, 
         incluindo o usuário e as suas próprias aplicações. \citep[tradução nossa]{Abowd:1999}
\end{quote}

Outras referências: \cite{Alex:2010}, \cite{Weiser:1991} e \cite{norell:thesis}.

\section{Objetivos}
O objetivo deste trabalho é .....

\section{Tabela}
Um exemplo de tabela é a \ref{tabrecursosdisp1}:



\begin{table}[!ht]
	\caption{Comparação entre recursos disponíveis}
	\begin{center}
		\begin{tabular}{ll|ll}
			\hline
			\multicolumn{2}{c}{Cacti}       & \multicolumn{2}{|c}{Zabbix}                   \\ \hline
			& CPU usage                     & CPU jumps                                  &  \\ \hline
			& CPU utilization               & CPU load                                   &  \\ \hline
			& Load usage                    & CPU utilization                            &  \\ \hline
			& Load averge                   & Disk space usage                           &  \\ \hline
			& Logged in users               & Memory usage                               &  \\ \hline
			& Memory usage                  & Network traffic on eth0                    &  \\ \hline
			& Ping Latency                  & Network traffic on eth2                    &  \\ \hline
			& Traffic (bits/sec)            & Swap usage                                 &  \\ \hline
			&                               & Value cache effectiveness                  &  \\ \hline
			&                               & Zabbix cache usage, \% free                &  \\ \hline
			&                               & Zabbix data gathering process busy         &  \\ \hline
			&                               & Zabbix internal process busy               &  \\ \hline
			&                               & Zabbix server performance                  &  \\ \hline
		\end{tabular}
	\end{center}
	\small{Fonte:Adaptado de \cite{dos2016comparativo}.}
	\label{tabrecursosdisp1}
\end{table}


\chapter{Desenvolvimento}

Este é o desenvolvilmento ...

\section{Seção1}
Este é um tipo de seção

\

\begin{figure}[!ht]
	\caption{Logo do Curso}
	\begin{center}
		\includegraphics[width=8cm,height=8cm]{imagens/logoREDES}
	\end{center}
	\small{Fonte: acervo pessoal.}
	\label{fig:redes}
\end{figure}

\

continua texto...
\subsection{Subseção1}
Este é um tipo de subseção

\subsection{Subseção2}
Este é um tipo de subseção

\subsubsection{Subsubseção1}
Este tipo de subsubsection

\paragraph{Seção quinária}
Este é um tipo de seção quinária


\chapter{Conclusão}

Está é a conclusão do trabalho ....

\setlength{\baselineskip}{\baselineskip}

%%=============================================================================
%% Referências
%%=============================================================================
\bibliographystyle{abnt}
\bibliography{referencias/referencias}



%IMPORTANTE: Se precisar usar alguma seção ou subseção dentro dos apêndices ou
%anexos, utilizar o comando \tocless para não adicionar no Sumário
%Exemplos: 
% \tocless\section{Histórico}
%%=============================================================================
%% Apêndices
%%=============================================================================
%\appendix
%\include{capitulos/apendicea}
%\include{capitulos/apendiceb}

%%=============================================================================
%% Anexos
%%=============================================================================
%\annex
%
\chapter{Título do Anexo}
Este é o anexo A



\end{document}
